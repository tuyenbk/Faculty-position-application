\documentclass[a4paper,11pt]{article}
\usepackage[margin=0.7in,top=0.7in,bottom=0.7in]{geometry}%for margin setting
\usepackage{lastpage} % for getting the number of the last page in the document

\usepackage{indentfirst} %first par not indented
\usepackage{setspace}
\usepackage{lineno}%for line number
\usepackage{sectsty}%for section format

\usepackage{fancyhdr} %for header/footer
\pagestyle{fancy}
\fancyhf{}
\lfoot{Teaching Statement - Tuyen Le}
\rfoot{Page \thepage\ of \pageref{LastPage}} 
\renewcommand{\headrulewidth}{0in}%remove underline in header

\usepackage{pslatex} %Times font

%-----------------GUIDE-------------------------
%purpose: reflect author's beliefs and practices about teaching and learning process
%content included:
%	- LEARNING: your conception of how learning occurs? What do you meaning by learning?
%	- TEACHING: how DID/WILL  you teach? What do you mean by teaching? how do you facilitate learning process?
%	- justification for why to teach in the way you do
%	- TEACHING EXPERIENCE & PLAN: the goals for yourself and your students
%	- your interests in new technologies, activities and types of learning
%format:
%	-1-2 pages
%-----------------END GUIDE-------------------

\begin{document}
\singlespacing
\setstretch{1}
\sectionfont{\centering}

\section*{\large Teaching Statement - Tuyen T. Le}

%---OVERVIEW
%
One of the primary goals of my faculty career is to educate the next generation of construction engineers to be not only well prepared with fundamental knowledge but also be able to adopt cutting-edge technologies in the rapidly changing field of construction. 
%
My passion in helping students effectively achieving these goals has shaped my teaching philosophy over the years working as a lecturer and instructor.
%
I believe that learning is a process of pattern recognition; and practicing in solving real problems of the industry would facilitate knowledge acquisition and skill development in construction engineering students. 
%Learning effectiveness would be even more significantly enhanced when students are engaged and actively involve in the learning process.
%
%implementation values of lessons, methods, knowledge and techniques.
%My years of experience in teaching has shape my teaching style to ensure that as a teacher, i will not not a knowledge transfer but the  fundamental knowledge, (2) build critical skills, and (3) find an effective method to support me achieve the above objectives. 
%
%---TEACHING EXPERIENCE
%teaching and mentoring experience, this experience form my belief and teaching style
\par
I have gained my teaching and supervising experience as a lecturer for more than two years (2011-2013) at Ho Chi Minh City University of Technology (HCMUT) and during my doctoral study at Iowa State University (ISU). 
%
At HCMUT, I taught a broad number of undergraduate courses ranging from {\sl Economics in Construction} to {\sl Construction Project Management}. I developed lecture notes, and designed in-class exercises, assignments and exams. I also supervised around 50 students each semester working in a term project course on construction method and planning. 
%
At ISU, I have been an instructor of an web-based graduate course on {\sl Construction Scheduling and Estimating} (CE594A) for three consecutive semesters since Fall 2015. I reviewed, revised and updated the library of test questions and answer students' questions by face-to-face meetings, emails or phone calls when they need further explanations regarding the module content or tests. Recently, I was invited to give a guest lecture on the implementation of  Natural Language Processing (NLP) in managing civil infrastructure data in CE595A which is about {\sl Research Method in Construction Management}. Presently, I'm supervising an undergraduate student on an funded project. I assign him specific tasks, provide guidance on how to collect data, and review and edit his written reports. 
%
%---TEACHING STYLE
%
\par
Through the above teaching experience, I find that template learning is an effective means to enable knowledge acquisition to occur in students, especially in the engineering field. To facilitate learning, I designed my lectures around in-class exercises, homework assignments and case studies. For each topic, I developed several exercises with different input settings to stimulate their interpretation ability and allow them to be able to solve new problems different from those provided in the class.
%
I have learned that students show their attention and actively involve in this process when they understand the practical value and implementation scenarios of engineering methods and techniques.
%
I usually start a lecture by introducing a real-world engineering problem and ask students to find the answer using their logics.  
%before teaching techniques or methods and explaining in detail their principles and logics behind.
%
For example, before introducing the CPM scheduling method, I designed and requested students to determine the completion date for two projects with different complexity degree. The first example, which is a simple single footing construction consisting of a few task, can be easily resolved. The other is a more complex process of constructing a basement. While solving this problem, I explained the underlining logics resulting proposed steps and formulas in CPM and illustrate how it can help construction engineers handle large projects. 
%
By doing that, students are able to understand the value of CMP in solving complex scheduling problems and consequently enhance their interests in learning. In addition, being familiar with the process of developing a new technique, my students are equipped with self-studying skills to tackle other engineering challenges. 
%learning technologies, images says more than words. add figures in my presentation, similuation construction to show.
%in class, remind what have learning the last class, any question, since students need sometime to digest the information, they may not have question at the end of the class, but after sometime review and digest knowledge they my discover something that do not understand.
%I have office hours every week so that students need help on what they have not fully understand. 
%construction is team-work based job. i will design homework, case studies which a team of three or four students need to work together
%
\par
I believe that the teaching goals should be not limited to providing students with engineering knowledge but also include introducing them with cutting-edge research and implementation technologies. In my future teaching, I plan to incorporate a case studies of utilization of advanced technologies (e.g., BIM-Building Information Modeling) for corresponding sections in my class. For example, I will incorporate 4D-scheduling into the section of scheduling techniques or 5D-BIM for cost estimating. This will ensure student have a holistic picture of the state-of-the-art and practices of the industry so that they can notice the required skills they need to be ready before starting their career. I will design some optional term projects as creative components with bonus points to encourage students to get early involved in research activities. 
%
%mentoring and supervsing style, i learned that, for undergradate students especially junior students, we need to micro supervising by providing detail guide, specific task, and deadline. weekly meeting with  

%---FUTURE PLAN 
\par
%
My teaching experience and interests range from the courses on construction project planning, including finance, scheduling and estimating, to construction method design. I can also examine relevant reference materials to be able teach other courses for both undergraduate and graduate levels of the construction engineering and management program. 
%
For the graduate level, I'm interested in offering a class on the implementation of data and information technology in construction and infrastructure management. This course will introduce students with BIM-aided project management techniques including 4D-scheduling, automated estimating, clash detection. A module on digital data project delivery will be designed to discuss the current practices regarding digital data utilization and transferring among project participants within construction projects. I plan to design the class in both traditional lecture and online formats. My research expertise and the experience in instructing an online class at ISU help me accumulate materials, knowledge and skills required to successfully design and run this class.

\end{document}
