\documentclass[a4paper,11pt]{article}
\usepackage[margin=0.8in,top=0.8in,bottom=0.85in]{geometry}%for margin setting
\usepackage{lastpage} % for getting the number of the last page in the document

\usepackage{indentfirst} %first par not indented
\usepackage{setspace}
\usepackage{lineno}%for line number
\usepackage{sectsty}%for section format

\usepackage{fancyhdr} %for header/footer
\pagestyle{fancy}
\fancyhf{}
\lfoot{Teaching Statement - Tuyen Le}
\rfoot{Page \thepage\ of \pageref{LastPage}} 
\renewcommand{\headrulewidth}{0in}%remove underline in header

\usepackage{pslatex} %Times font

%-----------------GUIDE-------------------------
%purpose: reflect author's beliefs and practices about teaching and learning process
%content included:
%	- LEARNING: your conception of how learning occurs? What do you meaning by learning?
%	- TEACHING: how DID/WILL  you teach? What do you mean by teaching? how do you facilitate learning process?
%	- justification for why to teach in the way you do
%	- TEACHING EXPERIENCE & PLAN: the goals for yourself and your students
%	- your interests in new technologies, activities and types of learning
%format:
%	-1-2 pages
%-----------------END GUIDE-------------------

\begin{document}
%\linenumbers
\singlespacing
\setstretch{1.0}

\sectionfont{\sectionrule{0ex}{0pt}{-1ex}{.75pt}}

\section*{Teaching Statement}
%---OVERVIEW
%
One of the primary goals of my faculty career is to educate the next generation of construction engineers who are not only well prepared with fundamental knowledge but also able to adopt cutting-edge technologies in the rapidly changed field of construction. 
%
My passion in effectively helping students achieving these goals has shaped my teaching philosophy over the years working as a lecturer and instructor.
%
I believe that learning is a process of pattern recognition; and practicing in solving real problems in the industry through in-class exercises, take-home assignments, case studies would facilitate knowledge acquisition and skill development in construction engineering students. Learning efficiency would be even more significantly enhanced/facilitated when students are engaged and actively involve in the learning process.
%
%implementation values of lessons, methods, knowledge and techniques.
%My years of experience in teaching has shape my teaching style to ensure that as a teacher, i will not not a knowledge transfer but the  fundamental knowledge, (2) build critical skills, and (3) find an effective method to support me achieve the above objectives. 
%
%---TEACHING EXPERIENCE
%teaching and mentoring experience, this experience form my belief and teaching style
\par
I have gained my teaching and supervising experience during the years when I worked for more than two years (2011-2013) as a lecturer at Ho Chi Minh City University of Technology (HCMUT) and during my doctoral program at Iowa State University (ISU). 
%
At HCMUT, I taught a broad number of undergraduate courses including {\sl Economics in Construction} and {\sl Construction Project Management}. I developed and delivered lectures, and designed in-class exercises, assignments and exams. I also supervised around 50 students each semester on the term project on construction method and planning. 
%
At ISU, I have been an instructor of an online graduate course on {\sl Consruction Scheduling and Estimating} (CE594A) for three consecutive semesters since Fall 2015. I have reviewed, revised and updated the library of test questions. I also invited to deliver a guest lecture on research methodology in CE595A which is about {\sl Research Method in Construction} In addition, I been supervising an undergraduate student on an funded project. I given him specific assignments and provided guidance on how to collect data, reviewed and edited his written reports. 
%
%---TEACHING STYLE
%
\par
Through the above teaching experience, I have recognized that acquisition of knowledge, especially in the engineering field, is effective through pattern learning. To facilitate learning, I designed my lectures around in-class exercises, homework assignments and case studies. 
%
I have also learned that students show their attention and actively involve in class activities when they understand the practical value and implementation scenarios of engineering methods and techniques.
%
In each lecture, I usually start with introducing a real-world engineering problems and ask students to solve it using their logics before teaching techniques or methods and explaining in detail their principles and logics behind.
%
For example, before introducing the CPM network scheduling method, I designed and have students to determine the completion date of two projects; one is a simple single footing construction process with a few tasks and clearly precedence relations, the other is more complex process of constructing a basement. Based on this examples, I explained the logics of steps and formulas in CPM and how it can help construction engineers handle large projects. 
%
By doing that, students are able to understand the value of CMP in solving complex scheduling problems and consequently enhance their interests in learning. In addition, being familiar with the process of developing tools and method, my students are equipped with self-studying skills to overcome new problems. 
%learning technologies, images says more than words. add figures in my presentation, similuation construction to show.
%in class, remind what have learning the last class, any question, since students need sometime to digest the information, they may not have question at the end of the class, but after sometime review and digest knowledge they my discover something that do not understand.
%I have office hours every week so that students need help on what they have not fully understand. 
%construction is team-work based job. i will design homework, case studies which a team of three or four students need to work together
%
\par
I believe that the teaching role of a faculty is not limited to provide students with engineering knowledge textbook but also include introduce of cutting-edge research and implementation technologies. For example, I plan to incorporate a case studies of utilization of BIM (Building Information Modeling) for corresponding sections in my class, such as introducing 4D-scheduling into the section of scheduling techniques or 5D-BIM for cost estimating. This will ensure student have a holistic picture of the state-of-the-art and practices of the industry so that they can know required skill they need to be ready before starting their career. I will design some optional term projects as creative components with bonus points to encourage students who have passion in research to be able to involved. 
%
%mentoring and supervsing style, i learned that, for undergradate students especially junior students, we need to micro supervising by providing detail guide, specific task, and deadline. weekly meeting with  

%---FUTURE PLAN 
\par
%
Although my teaching experience and interests are line with the courses of construction project planning (including finance, scheduling and estimating), and construction method; I will also be glad to examine reference materials to be able teach other courses for both undergraduate and graduate levels of construction engineering and management program. 
%
For specifically the graduate level, I'm interested in offering implementation of technology in design and construction engineering such as BIM that provide student example case and practial assignment which as student to use BIM-supported tools like Revit to do 4D-scheduling, or automated estimated, clash detection. I plan to offer this class under two sections that are campus and off-campus. I will designed an online course for distant students who are working in construction sites and save time commuting. Based on my previous research expertise, i plan to develop a new course on Data and Information Management and Utilization with the focus on infrastructure projects. 

\end{document}
