\documentclass[a4paper,11pt]{article}
\usepackage[margin=0.7in,top=0.7in,bottom=0.7in]{geometry}%for margin setting
\usepackage{lastpage} % for getting the number of the last page in the document

\usepackage{indentfirst} %first par not indented
\usepackage{setspace}%for space between lines
\usepackage{lineno}%for line number
\usepackage{sectsty}%for section format

\usepackage{fancyhdr} %for header/footer
\pagestyle{fancy}
\fancyhf{}
\lfoot{Research Statement - Tuyen Le}%footer
\rfoot{Page \thepage\ of \pageref{LastPage}} %page
\renewcommand{\headrulewidth}{0in}%remove underline in header

\usepackage{pslatex} %Times font


%--------GUIDE--------
%?	1-2 page, 3 at most, mostly not more than 1.5 pages, unless gotten other advices from committee
%?	use informative section headings and subheadings, or bullets
%?	indicate writing ability, technical but readable to all members even for whom outside the field
%?	provide a big picture focus
%?	provide summary, background, context/relevance/significance of my research
	%o	research interests
	%o	background
	%o	key problem to be resolved
	%o	past and current research and plan
%?	provide examples (pas and current research) to illustrate research efforts to address questions under the selected topic umbrella
	%o	value/significance of research
	%o	major findings/preliminary results/initial accomplishments/outcomes
	%o	publications/applications in prestigious journals (Autocon, JCCE-ASCE, etc.)
%?	short-term future research 
	%o	research limitations in past and current research
	%o	potential sponsors
	%o	research significance
	%o	ability to secure grant, proof by funding history 
%?	long-term research goals/research 
	%o	 4-5 topics/proposals/potential funding partners (NSF, States and Federal DOT), industrial collaboration (software venders)
	%o	why research is important
	%o	Feasibility of proposals, preliminary results, experience, or provide evidence from the past successful proposals for similar research to highlight the possibility to secure grants
	%o	Future research built upon current and the past research
	%o	interdisciplinary collaborators
	%o	including potential impacts and outcomes
	%o	benefits to the target university, students and schools (students involved, new courses developed)
%?	use language/story to demonstrate independence as a researcher (research ideas, ability to earn grant money, funding history, prestigious journal articles-Autocon, JCCE)
%?	identify major problems/questions and their relevance to the main research themes and significance to the field
%?	no need to list as many as possible future research, every one knows that I will work more than what are mentioned here
%?	tailor to the needs of the target department and position 
%prepare a longer version for interview (budget, equipment, graduate students) to start up the research
%-----------------END GUIDE-------------------

%-----------------VOCABULARY---------------
%-----------------END VOCABULARY--------

\begin{document}
%\linenumbers
\singlespacing
\setstretch{1}
\sectionfont{\centering}

\section*{\large Research Statement - Tuyen T. Le}
My research interests are in line with data and information technologies, and intelligent systems to enhance life-cycle data utilization in construction and infrastructure management. 
%
The adoption of advanced digital technologies such as Building Information Modeling (BIM), Geographic Information System (GIS), or Lidar throughout the project life-cycle has enabled a large portion of asset data to become available in digital format. 
%
The improved efficiency in sharing and utilizing of these complex machine-readable datasets, will in turn, translate into increased productivity, cost savings, and efficiency in project delivery and accountability.
%
%My primary career goal is to add value to the existing isolated data to reduce manual and duplicate efforts in data collection, processing, and utilization by introducing new knowledge, computational intelligence infrastructure and methods that can support automated data extraction and mining.
%
In the last three years of my doctoral program, I have conducted ample research on reducing manual and duplicate efforts in data collection, processing, and utilization by creating new knowledge and computational infrastructure with regard to 1) life-cycle data linkage, 2) data terminology discrepancy, 3) natural language based data extraction, and 4) data and information flow through transportation assets' life cycle.
%
In my future research, I plan to continue to work on facilitating seamless digital project delivery and asset management and broaden my research focus to the areas of intelligent systems and mining texts, sounds, visuals and other unstructured data.
%
%These data are presented in human natural language which is non machine-readable formats such as texts, and potential information sources in visual media like videos, images. 

%However, due to the fragmented nature of the construction industry, digital data sets are being archived, and managed in a non-collaborative manner. Due to this reason, the transportation assets as a whole have not yet fully benefited from the growing amount of digital data since digital data from distinct actors are yet to be linked and fully reused.

%However, due to the fragmented nature of the highway industry sector, digital data sets are being archived, and managed separately and in their own preferred formats. Due to this reason, the transportation assets as a whole have not yet fully benefited from the growing amount of digital data since digital data from distinct actors are yet to be linked and fully reused and external actors are not aware the existence or value of other data inventory. Since different project participants may use proprietary software platforms with different data structures, exchange of data becomes very challenging. Data exchange in a heterogeneous environment may lead to data loss, damage and requires time consuming processing in downstream phases.

%The major cost was time spent finding, verifying facility and project information, and transferring that information into a useful format. 

%By seamlessly using electronic engineered files generated during planning, design and construction phases, a significant amount of efforts can be saved as assets are managed in order to provide superior results.
\par
{\bf Life-cycle data space linkage}. In the conventional practice, life-cycle data are archived and managed individually by different departments in isolated and heterogeneous data warehouses within a highway agency.
%
To enable decision makers in asset management to effectively reuse digital data created by upstream design and construction partners, I developed a data linking platform that interconnects disparate digital data sources (BIM models, construction and project and asset management systems).
%
The system includes several translators to convert data into graph-based network where relevant data items are linked to one another across the data inventory. 
%
%The findings from this study are expected to provide an effective and efficient means to facilitate seamless digital data exchange throughout the life cycle of a highway project. 
%
I plan to expand this work from management of a single asset to integrated urban and infrastructure management to allow for concurrent collaboration between construction sectors (building, pipeline, railway, water supply, etc.).
%
Once local datasets become accessible to other related disciplines, better decision making with holistic and long-term benefits would be achieved.

\par
{\bf Inconsistency of data terminology.} Data terminology discrepancy is a big hurdle to integration or sharing of digital data among multiple disciplines, partners, or geographic regions. 
%
The lack of common understanding to the same or similar data presented in different terms can lead to the extraction of wrong data or misinterpretation. 
%
%The inconsistency of data terminology due to the fragmented nature of the highway industry has imposed big challenges on integrating digital data from distinct sources. 
%
%The issue of semantic heterogeneity may lead to the lack of common understanding of the same data between the sender and receiver. Explicit semantic relations among terms in digital dictionaries, such as ontologies can enable the meaning of a roadway concept name to be transparent and unambiguously understood by computer systems. However, due to the lack of an effective automated method, current practices of identifying these relations hardly rely on a manual process of knowledge acquisition from domain experts or text documents which is laborious and time-consuming.
%
%The lack of common understanding of the same data represented in different terms leads to the failure of data exchange or the extraction of wrong data. This proposed research will provide the civil infrastructure industry (specifically, the highway industry) with a powerful algorithm that will be based on the recent advancements in Natural Language Processing (NLP) to recognize users intention from their natural language input. A semantic matching algorithm will be designed to enable the automated extraction of the entity names having meanings equivalent to the users desired data.
%
%The diversity of data terminology is a big hurdle to the computer-to-computer communication when computer is not yet able to precisely understand data meaning; data integration in such a heterogeneous environment is therefore likely lead to failure or extraction of wrong data. This unresolved issue has turned into a big burden to end users who still play as a middleware in digital data exchange; and in many cases, it is impossible for them to deeply understand the structure and meaning of data labels stored in large and complex datasets to precisely extract subsets of data.
%
To enable semantic transparency for commonly used technical terms among highway agencies across the United State, I have developed a computational infrastructure that supports automated development of a machine-readable dictionary of American-English civil engineering terms. 
%
The proposed platform leverages Natural Language Processing (NLP) techniques and Artificial Intelligence (AI) to extract roadway terms and learn their meanings from roadway design manuals, guidelines and specifications.
%
%This NLP based methodology is expected to assist professionals in extracting roadway terms and their semantic relations from text documents.
%
%The present framework is expected to significantly reduce manual efforts and become an enabling tool that can help researchers in the highway domain quickly develop supporting ontologies and other forms of semantic resources for their specific use cases. 
%
I have developed an algorithm to classify heterogeneous technical terms into distinct semantic groups that are synonyms, hyponyms, and attributes.
%This would accelerate the process of removing the current bottleneck in machine readable dictionaries which are required for an unambiguous data sharing, integration or exchange.
%
In future research, I'm interested in implementing the proposed method in developing a national map of term synonyms among highway agencies which is crucial to avoid mismatches when integrating state historic project data (e.g., cost index, asset condition) to support data-driven infrastructure management. 
\par
{\bf Natural language based data retrieval engine.} Regarding to data acquisition, I have recently developed a successful proposal (PI: Dr. Jeong) which is awarded for mostly \$300,000 by National Science Foundation (NSF) to develop a computational partial model extraction engine that allows users to use plain English data requirements to query civil infrastructure digital data.
%
Simplicity in acquisistion of desired data from large and complex machine-readable digital infrastructure data critically decides the degree of reusability of up-stream digital models and their associated project data. 
%
However, in the traditional ad-hoc data retrieval approach, extracting digital data, which is a big burdens on professionals, is manual, time-consuming and error-prone. Users are required to have deep understanding of data structure, meanings behind each data label and a query language. 
%
This research aims to develope an automated data retrieval engine which is capable of recognizing user intention from their natural language queries (e.g., words, phrases, questions) and extracting the desired data from heterogeneous digital datasets.
%
I'm currently a lead researcher in an interdisciplinary project team of Linguistics, Machine Learning and Civil Engineering. 
%
In order to enable computer systems to understand users data requirements in natural language, I have been translating domain knowledge in design manuals, guidelines and specification into an extensive machine-readable dictionary for the civil infrastructure using my recently developed NLP-based method as mentioned above.
%
Upon completion of this digital dictionary, I will utilize NLP to develop an algorithm for interpreting and translating users' natural language inputs into machine-readable query codes. 
%
%An un-supervised machine learning model will be applied to build up a knowledge base that will consist of formal definitions of technical terms in the civil sector, specifically the highway industry sector.
%
%The direct intellectual merit of this proposed research is to address the fundamental interoperability issue in digital data exchange and provide the civil infrastructure industry with a user-friendly and automatic data extraction platform. 
%
%This research is expected to make transformative impacts on digital date exchange and sharing in the civil infrastructure industry, and promote and accelerate the industry transition to the digital project delivery as digital models and their associated data can be readily and seamlessly reused through the project life cycle.
%
%This research will fundamentally transform the way data users interact with and query digital modeling data and information in the civil infrastructure domain.
%
I plan to develop a NSF proposal expanding this data retrieval system to speech recognition that allow users to communicate with BIM/CIM models using their voice.
%
%The research outcomes will provide fundamental tools and resources for other researchers and industry professionals for various text mining and intelligence inference systems which are emerging research areas in the construction industry. Thus, this research will create significant synergistic and ripple effects to the construction industry.
%
%In order to fill that gap, this research proposes a novel approach for a fast and unambiguous reuse of digital models for the civil infrastructure industry by developing an automated data retrieval engine which is capable of recognizing user intention from their natural language queries (e.g., words, phrases, questions) and extracting the desired data from heterogeneous digital datasets. This research will employ the recent advances in Natural Language Processing (NLP) techniques, machine-learning based semantic measure methods to develop the data retrieval system. NLP will be utilized to process and interpret users natural language inputs. An un-supervised machine learning model will be applied to build up a knowledge base that will consist of formal definitions of technical terms in the civil sector, specifically the highway industry sector. A knowledge base, also known as a digital dictionary, is a critical component for intelligence systems to perform such knowledge works as interpretation of meaning behind human-oriented inputs. A semantic data retrieval algorithm will be designed to match the user intention to the data entities in the heterogeneous source dataset. The matching will be based on the meaning equivalence rather than the string similarity to eliminate the potential of false extraction due to the issue of data terminology inconsistency between the data creator and the user. 
%
%This research is envisioned to fundamentally transform the way in which professionals interact with complex and non-human-readable digital datasets in the civil infrastructure industry. This project will provide a novel computational infrastructure that will enable users to express data queries in plain English. If successful, the burdens currently imposed on users will be significantly eliminated. Once data extraction from digital models becomes straightforward, the bottle neck of MVD availability is also expected to be removed. The research success will translate into the willingness of accepting digital datasets by all project stakeholders, and the seamless digital data exchange through the project life cycle can be truly achieved.
%
\par
{\bf Life-cycle Data and Information Flow.} In the spectrum of implementation research, a research proposal mainly contributed by myself is funded by the Iowa Highway Research Board and Mid-West Transportation Center for \$180,000 to enhance the understanding of data and information workflow during the life-cycle of transportation assets.
%
To accomplish this objective, I have been conducting a series of working group discussions with professionals from different divisions and disciplines involved during the life-cycle for various type of transportation assets including signs, guardrails, pavements, etc.
%
Using the knowledge captured from these discussions, for each type of asset, I have developed several process maps visualizing the current data workflow and a data exchange matrix specifying when and what data required to be shared by whom and to whom.
%
%For each type of transportation assets, I have developed current practice and ideal process map and exchange requirement documents that show what data, who and when to be transferred to whom.
%
%The significant improvement of data and information sharing between project participants and across various project development stages is possible with a model based project delivery process, and electronic and digital data transfer systems, which will in turn translate into increased productivity, efficiency in project delivery, accountability, and asset management.
%
%The proposed research aims to develop a guide to help professionals of State Departments of Transportation (DOTs) understand the digital data and information flow during the project life-cycle for various type of transportation assets including pavements, bridges, culverts, signs, guardrails, etc. The guidebook will include but not limit to the following topics: (1) business use cases in which data sharing between project actors is needed, (2) business processes that define clear sequences of the activities to be performed for data and information sharing and exchange, and expected outcome; and (3) data requirements, data sources, levels of detail, software applications and tools involved in specific data exchange use cases.
%
%The objective of this study is to capture industry experts knowledge and needs regarding digital data and information sharing during the life cycle of transportation assets. In order to achieve that goal, Literature review, benchmarking the vertical industry practices, and focus group discussions will be extensively used. A working group will be formed including domain industry professionals with various expertise from State DOTs, consultants, contractors, and software vendors. This focus group discussion will help identify and document the data exchange scenarios, data flows, data requirements, data format, and supporting software applications. Based on discussion results, a process map and a data map will be developed. The process map will show the data exchange processes throughout the project lifecycle, and the data map presents what data required to be shared by whom and to whom and when. The next section specifically describes required work tasks.
%
%This project is expected to provide a better understanding on data and information flow throughout the lifecycle for various transportation assets. 
This understanding of the current life-cycle data flow will allow researchers and professionals to identify current roadblocks in digital data transferring, and provide recommendations for leverage existing data to reduce waste in data-recreation. 
%
In my future research agenda, I plan to develop a research proposal targeting NCHRP that aims to develop a national guide that can be used by highway agencies to evaluate their maturity in digital based project delivery, identify requirements and tools that need to help them facilitate seamless digital data transferring throughout the project life-cycle.  
%
%The major deliverables of this project include process maps, data maps and a guidebook for DOTs that can facilitate the implementation of data and information transfer for highway asset management.
%
%Value of future research
\par
{\bf Future research.} In addition to continuing my research on data sharing and retrieval as discussed above, my long term goals are smart building and civil information model (smart BIM/CIM) and big data analytics. Smart BIM/CIM is my vision for the next generation of digital models which does not contain only data and information but is also integrated with domain knowledge which can perform self-reasoning and response to the {\it what if} question to assist in design, project planning and in other decision making.
%
I plan to utilize NLP to derive and translate domain knowledge in text documents such as specifications into machine-readable rules, and integrated these digital knowledge into BIM/CIM models to create smart digital models. 
%
In addition, I'm also enthusiastic with pursuing my research career goals in the area of data mining from texts, visuals or other types of unstructured data.
%
Unstructured data, particularly text documents such as project contracts, RFIs (Requests for Information), project inspection reports are still play a major role in data and information sharing and communication among project stakeholders.
%
Manually reading, reviewing and analyzing these texts are laborious, time-consuming and error-prone. 
%
I'm interested in developing computer-assisted text reader systems to assist researchers and professionals in reviewing project related text documents and digging for further value information.
%
My long-term research focus also include alternative data and information source such as visuals (images, videos) or sounds to reduce data collection efforts for decision making in construction job site and infrastructure management.
%
%Constructing BIM information models using current method such as Lidar are still have low accuracy and time-processing for those is costly.
%
%One-call center data is just message transferring to participant operators, in urban cities where numerous underground facilities utilities such as electricity, water pipe and sewer, gas, internet, cables using natural language speed, excavator, location, locates, excavating, digging.  One-call center receive request and send it to participant members who are operators and owners of facilities who is responsible for locate/marking their utilities by paints or flags. Research is need to allow for construction of a digital library of those utilities. Average speed to answer takes hours and days.
%
%Moreover, Im also every interested in intelligence project delivery, virtual reality and intelligence system in design and reasoning system where user interact with components of project (design, planning, decision making, cost scheduling) using natural language. The system can answer what if questions in design, planning of the project. Images and video processing the keep track worker health condition and warning, alert when unusual.
%
%For educational research project, construction knowledge for instance, construction activities in foundation construction for collecting for learning, where new knowledge and technology, these knowledge has been introduced and available online, a unique system that can, support self-studying construction engineering for both undergraduate and graduate students. This knowledge resources is important for practices in the preliminary design phase when historic design, construction data are value resources for AI (Artificial Intelligence) in learn and predict design, schedule, budget, risks and other aspects of project management. 
%
The success from my previous NSF proposal related to NLP implementation in CIM illustrates promising funding opportunity from national research programs. NSF and NIST (National Institute of Standards and Technology) are some examples of primary potential sponsors for my future research. Besides, being involved in several DOT projects, I have recognized that highway agencies show high interests in transitioning to digital data project delivery. I plan to secure funding from Federal Highway Administration (FHA), National Cooperative Highway Research Program (NCHRP) and State DOTs.
%
%Final paragraph  overall good expressions of my research
\end{document}
